\documentclass{article}
\usepackage{ulem}\usepackage{amsmath}
\usepackage[colorlinks=true, linkcolor=blue, urlcolor=blue]{hyperref}
\begin{document}
\begin{center}
\begin{table}[h!] \begin{center} \begin{tabular}{|c|c|} \hline       yolo&yolo    \\ \hline     yolo&yolo   \\ \hline     yolo&editor   \\ \hline  \end{tabular}\end{center} \end{table}
\begin{table}[h!] \begin{center} \begin{tabular}{|c|c|} \hline  &  \\ \hline  &  \\ \hline  \end{tabular}\end{center} \end{table}
    \begin{tabular}{ c c c }
        cell1 & cell2 & cell3 \\
        cell4 & cell5 & cell6 \\
        cell7 & cell8 & cell9
    \end{tabular}
    New textfield
\end{center}
\begin{table}[h!]
    \begin{center}
        \begin{tabular}{ | c | c | c | }
            \hline
            cell1 & cell2 & cell3 \\ \hline
            cell4 & cell5 & cell6 \\ \hline
            cell7 & cell9 &       \\ \hline
        \end{tabular}
    \end{center}
    \caption{To jest podpis tabeli}
\end{table}
\end{document}